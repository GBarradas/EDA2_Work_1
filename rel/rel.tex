\documentclass{article}
\usepackage[portuguese]{babel}
\usepackage[utf8]{inputenc}
\usepackage{makeidx}
\usepackage{graphicx}
\usepackage{fancyhdr}
\usepackage{enumerate}
\usepackage[a4paper, total={16cm, 24cm}]{geometry}


%---------------------------------------------------------------%
\fancypagestyle{indice}{
\fancyhf{}
\lhead{\includegraphics[scale=0.3]{imagens/logo.png} }
\rhead{U.C. Estrutura de Dados e Algoritmos II\\
\textbf{1º Trabalho-Mosaics}}
}
%---------------------------------------------------------------%
\pagestyle{fancy}
\fancyhf{}
\lhead{\includegraphics[scale=0.3]{imagens/logo.png} }
\rhead{U.C. Estrutura de Dados e Algoritmos II\\
\textbf{1º Trabalho-Mosaics}}
\rfoot{\thepage}
\setlength{\headheight}{1.5cm}
%---------------------------------------------------------------%
\title{ \includegraphics[scale=0.3]{imagens/uevora.png}\\
U.C. Estrutura de Dados e Algoritmos II\\
\textbf{1º Trabalho-Mosaics}}
%---------------------------------------------------------------%
\author{
\textbf{Docente: }Vasco Pedro\\
\textbf{Grupo: } g121\\
\textbf{Discentes: } André Baião 48092, Gonçalo Barradas 48402\\
}
\date{Março 2022}
%---------------------------------------------------------------%

\begin{document}

\maketitle
\clearpage
%---------------------------------------------------------------%
\renewcommand{\contentsname}{Índice}
\thispagestyle{indice}
\tableofcontents
\clearpage
\setcounter{page}{1}
\section{Descrição do algoritmo}

O algoritmo para resolver este problema consiste esta dividido em 2 partes.
A primeira parte consiste em analisarmos  linha a linha e identificar sequências de cores iguais.
A segunda parte  é responsável por calcular de quantas maneiras possíveis se consegue fazer uma sequência usando as peças de lego

\section{Função Recursiva}
TAMOS FUDIDOS QUE EU TAMBEM NAO SEI

\section{Calculo da Complexidade}
OUTRA FODA
\subsection{Temporal}
\subsection{Espacial}
\section{Comentários}
ACHO QUE ESTAMOS FUDIDOS
\section{Conclusão}
CONCLUINDO ESTAMOS FUDIDOS

\end{document}
